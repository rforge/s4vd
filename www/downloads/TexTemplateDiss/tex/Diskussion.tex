\chapter{Discussion}
In this paper we propose a new biclustering algorithm which combines the SSVD algorithm suggested by Lee et al. (2010) with the stability selection of Meinshausen and B\"uhlmann (2010). In brief, the model selection based parameter tuning of the penalized regression models of the SSVD algorithm is replaced by a subsampling based variable selection that controls type-one error rates. The  S4VD approach here presented allows to control the \textit{degree of sparsity} of the resulting SVD-layers by choosing desired type-one error levels. The stability selection estimates the selection probabilities of the rows and columns to belong to a bicluster. %These selection probabilities are an additional information supporting the interpretation of the clustering results. 
Depending on the chosen type-one error levels the results are robust biclusters represented by rows and columns that have high selection probabilities. If the noise level is getting too high the stopping criterion leads to an interruption of the S4VD algorithm preventing from fitting additional SVD-layers that correspond to noise. %In addition, due to the higher \textit{degree of sparsity} of the resulting SVD-layers, a biological interpretation is more straightforward compared to the SSVD algorithm. 
So far, the S4VD method is the only biclustering approach that takes the cluster stability regarding perturbations of the data into account. 

We applied the S4VD algorithm to evaluate a lung cancer microarray data set and showed that the resulting biclusters represent tumor subclasses together with coregulated genes. Genes that were known as markers, showed high selection probabilities in the respective biclusters. A gene set enrichment analysis revealed that the genes associated with identified biclusters belong to significantly enriched cancer related biological processes. In a simulation study the S4VD algorithm was compared with the SSVD algorithm, the improved Plaid Model (Turner et al., 2005) and the ISA (Bergmann, 2003). The S4VD algorithm showed the best performance regarding the recovery of biclusters and was less susceptible to noisy data compared to the other methods. 

However, the subsampling steps of the stability selection make the S4VD algorithm computationally demanding. We presented a simple improvement that strongly reduces the computation time. 