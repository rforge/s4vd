\chapter{Introduction}

\section{The microarray technology}
%\section{History}
Since Mendel 
Darwin -> Mendel -> DNA -> Helix -> PCR -> Sanger Sequenzierung -> Microarrays expression protein chip on chip sage cgh snp -> qPCR -> RefSeq Highthroughput -> measure epigenome transcriptome proteome genome
highdimensionl molecular data.
the curse of dimensionality
development of statistical methods  -> classification supervised differentially expression clustering
in clinical cancer research also clinical data
integration by means of interactive graphics

\section{Clustering} % Data Tables two way two mode clustering
Clustering methods belong to the most commonly used statistical tools in the analysis of high dimensional data sets. If additional information about the sample class labels is lacking, other types of analysis like supervised classification methods or testing for differentially expressed genes can not be performed. In this case unsupervised clustering allows to reveal unknown structures that are possibly hidden in the gene expression data matrix. These structures may be characterized by groups of genes that are coregulated by a common transcription factor and thus belong to the same pathway or samples that share a similar gene expression pattern. 
%Classical clustering methods are hierarchical clustering or k-means clustering. 
One disadvantage of commonly used clustering algorithms like hierarchical clustering or k-means clustering is that the cluster assignment of objects are based on the complete feature space, e.g. in case of clustering the samples the resulting clusters are derived with respect to all genes. 
But groups of genes may only be coregulated within a subset of the samples and samples may share a common gene expression pattern only for a subset of genes. Such clusters that exist only in a subspace of the feature space can hardly be detected by these classical one-way clustering algorithms.
To find such clusters other clustering concepts are needed. 

\section{Biclustering} %high dimensional data, also interesting in other fields amazon data, market segmentation, text mining, information retrieval
In the past decade, the concept of biclustering has emerged in the field of gene expression analysis. Biclustering which is also known as coclustering or two-way clustering describes the simultaneous clustering of the rows and the columns of a data matrix. 
The first biclustering algorithm, the so called Block Clustering, has been developed by \citet{Hartigan1972}. \citet{Cheng2000} proposed the first biclustering algorithm for the analysis of high dimensional gene expression data.
Since then, many different biclustering algorithms have been developed. 
Currently, their exists a diverse spectrum of biclustering tools that follow different strategies and algorithmic concepts.
Among others, popular algorithms are the Coupled Two-Way Clustering (CTWC) by \citet{Getz2000}, Order Preserving Sub Matrix (OPSM) algorithm by \citet{BenDor2003}, FLOC by \citet{Yang2003}, Spectral biclustering by \citet{Kluger2003}, xMotif by \citet{Kasif2003},
the Iterative Signature Algorithm (ISA) by \citet{Bergmann2003},
the Plaid Model by \citet{Lazzeroni2000} and the improved Plaid Model \cite{Turner2005}, SAMBA by \citet{Tanay2004}, biclustering by non-smooth non-negative matrix factorization by \citet{Carmona-Saez2006}, the Bi-correlation clustering algorithm (BCCA) by \citet{Bhattacharya2009} and factor analysis for bicluster acquisition (FABIA)\cite{Hochreiter2010}. \citet{Prelic2006} developed a fast divide-and-conquer algorithm (Bimax) and conducted a systematic comparison of different biclustering algorithms.
\citet{Santamaria2007} published an article on validation indices for the evaluation of biclustering results and the comparison for biclustering algorithms. Comprehensive reviews about the concept of biclustering and the different biclustering approaches have been written by \citet{Madeira2004} and \citet{Mechelen2004}.

In a more theoretical review \citet{Busygin2008} emphasized the mathematical concepts behind several biclustering algorithms and pointed out that the SVD represents a capable tool for finding biclusters. Furthermore, most existing biclustering algorithms use the SVD directly or have a strong association with it. To keep track of the huge diversity, regarding the mathematical properties of the existing biclustering algorithms, Busygin et al. (2008) suggest to relate new and existing biclustering algorithms to the SVD. 

A major drawback of many biclustering methods is that they rely on random starting seeds and thus are inconsistent and results may vary even when the algorithm is applied to the same data set. As often in unsupervised clustering it is difficult to judge the biclustering results regarding their stability. For one-way clustering several resampling approaches to validate the stability of the clustering results are known, e.g. %bootstrap clustering (Kerr and Churchill, 2001)\nocite{Kerr2001},
multiscale bootstrap hierarchical clustering \cite{Suzuki2006} and consensus clustering \cite{Monti2003}. In case of biclustering, similar methods that take the stability of the results into account are not yet available. 

% Types of biclusters
